\documentclass[11pt]{article}
\usepackage{geometry}
% Math formatting
\usepackage{amsmath}
\usepackage{amssymb}

% Lay out packages
%\usepackage[margin=3.5cm]{geometry}
\usepackage[utf8]{inputenc}
%\usepackage{mathpazo}

% Dutch style of paragraph formatting, i.e. no indents. 
% \setlength{\parskip}{1.3ex plus 0.2ex minus 0.2ex}
% \setlength{\parindent}{0pt}

% Command for Horizontal lines
\newcommand{\HRule}{\rule{\linewidth}{0.5mm}}
% Command for degree symbol
\newcommand{\degree}{\ensuremath{^\circ}}

% Add images and pdf pages
\usepackage{graphicx}
\usepackage{pdfpages}

% Colored rows in tables
%\usepackage[table]{xcolor}

% Clickable links with hyperref package
\usepackage[pdfborder={0 0 0 0}, linkcolor=black, urlcolor=blue]{hyperref}

% Fancy Header
\usepackage{fancyhdr}
\pagestyle{fancy}

% Psuedo code
\usepackage{algorithm}
\usepackage{algorithmic}

\rhead{\large\bfseries Gamification and Learning Analytics for Education} % right header: document title
\lhead{\textsc{Jurriaans}, \textsc{Latour}} % left header: document author
\cfoot{\large \thepage} % center footer: page number

\setlength{\headheight}{18pt}

\begin{document}

\begin{titlepage}
\begin{center}
\includegraphics[width=1\textwidth]{img/uva}\\[1cm]
\HRule \\[0.4cm]
% Title
Literature Research on Applying Gamification and Learning Analytics to Education \\\small \small{[Version 0.1]}\\[0.4cm]
\HRule \\[1cm]
\begin{tabular*}{0.95\textwidth}{@{\extracolsep{\fill}} l r}
Robrecht \textsc{Jurriaans} & Sander \textsc{Latour} \\
\end{tabular*}

\vfill \today
\end{center}
\end{titlepage}

\newpage
\thispagestyle{empty}
\mbox{}
\pagebreak

\pagestyle{empty} %get rid of header/footer for toc page
\tableofcontents
\cleardoublepage %start new page
\pagestyle{plain} % put headers/footers back on
\setcounter{page}{1} %reset the page counter


\section{Introduction}
% Introduce the current status of education and the problems that lie with motivation
There is an ongoing desire to improve study success in higher education.
Work has been done to identify students that have a high risk of dropping out. (refs)

Two factors in study success seem to be 1) the match between the level of the student and the level of the courses' materials and 2) the student's engagement and motivation.

Students often come from different backgrounds and 

\cite{Horizon2012}

% Use psychological evidence to determine key-factors in motivation and how it improves learning


% Introduce Gamification and Learning Analytics as solutions to this motivation problem



% Research Questions


% Overview of Paper

\section{Gamification}
Games are unprecedented in motivating and engaging players. Humans tend to spent large amounts of time playing games of all sorts. This led some people to believe that it could be possible to generate this type of motivation and engagement in other contexts as well by applying the underlying game design patterns to areas other than gaming to steer behaviour.

Gamification is a concept that is both controversial \cite{McGonigal2011} as well as hard to define \cite{Deterding2011}. The controversy of the term gamification is easily seen in the many different terms that have been coined, including pervasive games, alternate reality games, productivity games, applied gaming, etcetera. All of these terms refer to the practice of applying game design patterns to non-gaming contexts to achieve higher motivation and engagement from users. The term gamification originates from 2004 when it was first used by Nick Pelling as a marketing term \cite{Huotari2012}. In recent years, gamification has moved beyond being just a marketing technique and becoming an academic field. The mayor problem with gamification is that the definition is very subjective and it is difficult to formalize the definition \cite{Huotari2012, Deterding2011}. 
%
The definition of Gamification given by Deterding is:
\begin{quote}
	``Gamification'' is the use of game design elements in non-game contexts. 
\end{quote}

The two important concepts in this definition are ``game design elements'' and ``non-game contexts.'' ``Game design elements'' or ``game atoms'' \cite{Deterding2011, Brathwaite2008} are elements that are not unique to games, but instead are used often in ``gaming contexts''. Such elements can be the state, the dynamics, theme, goals and mechanics of a game and the player and his avatar. The definition for game elements is also subjective as any of these elements can be used outside of the game-context or ``magic circle of games'' \cite{Huizinga}. To give a formal definition of game-elements we would first need to define game-contexts.

Another definition for gamification is given by Huotari et al \cite{Huotari2012} which is from a marketing point of view.
\begin{quote}
	“(Gamification is a) service packaging where a core service is
enhanced by a rules-based service system that provides feedback
and interaction mechanisms to the user with an aim to facilitate
and support the users’ overall value creation.”
\end{quote}

Deterding challenges this definition by addressing that a rule-based system covers more than just games \cite{Deterding2011}. Furthermore, by viewing gamification only from a service marketing perspective does not fully take the social and experiential dimensions of games into account. Nevertheless, both definitions see gamification as the introduction of game-like elements into non-gaming contexts.

Non-game contexts is a broad term for anything that is not a game in itself. However, non-game contexts may also apply to games themselves \cite{Deterding2011}. Parts of the game that may not necessarily be part of the design of the game can be considered as non-game contexts. These type of meta-games have been considered as separate from the game design itself. The game-context is thus also difficult to capture in a formal definition as games can be considered to be something subjective by nature. 

In game theory, games are formally defined as a set of players or agents, a set of moves or strategies and a pay-off function or reward system. In this definition the game elements are already considered as being the set of moves and the reward function. The reward function would then hold all the underlying game mechanics whereas the set of moves can be interpreted as the user interface. For a gamification framework we would like to be able to understand more about the elements that fall under reward functions and the set of moves, as there is common notion that gamification is more then adding rewards and available moves to a non-gaming context.

\subsection{Game Elements}
Byron and Reeves \cite{Byron2009} consider game-elements to be any part of a game and give a list of game elements that help to make a game more engaging. These game elements are: Self-representation with avatars; three-dimensional environments; narrative context; feedback; reputations, ranks, and levels; marketplaces and economies; competition under rules that are explicit and enforced; teams; parallel communication systems that can be easily configured; time pressure. These elements differ from the set of moves and the reward function as found in game theory. Although elements such as feedback and ranks seem to be correlating to intrinsic rewards and thus the reward function, avatars and communication systems appear to be different from this as they are actions that do not directly influence the pay-off.

% Ontology project
Another way of viewing game contexts and game elements is by creating an ontology such as the Game Ontology Project (GOP) \cite{Zagal2008} which is a framework for describing, analysing and studying games. It is a hierarchy for defining games using prototype theory and grounded theory to account for a growing and changing field of games. The ontology is based around four main categories: Interface; Rules; Entity Manipulation and Goals. The interface category comprises how a player interacts with the game, which comprises of both perception and user input. The rules are the rules in the game that work apart from the player, such as the physics, as well as direct consequences of user actions. This latter category seems to be related to the category of entity manipulation, but the main difference is that in the rules category, user actions affect certain underlying mechanics, whilst user actions affecting entities within the rule system belong to entity manipulation. The final category goals holds all the 
information about the direction of the game: when the game ends; how the score is calculated and the actual goals for the player.
These game elements are on a higher level than the game elements as described by Byron and Reeves \cite{Byron2009}, but are on a lower level than the game theory description of games. This ontology is based on the notion that game elements represent smaller chunks of gameplay \cite{Zagal2009} and thus the Game Ontology Project is mainly focused on gameplay elements.

% Game classification elements
A slightly different ontology is given by Elverdam \cite{Elverdam2007} which is based on the Typology Model of games. It is mainly used for the classification of games and works by separating the general concepts: Physical Space; Virtual Space; Internal Time; External Time; Player Relation; Player Composition; Struggle and Game state. Although the model is used for classification as opposed to generation or game design, it is still capable of describing the various game elements. It is a model which is geared towards separating the various types of game design patterns without giving actual separate game elements, but rather categories of game elements which are present in game-play. The eight different categories can be categorised on a higher level as: Players; Space; Time and Rules. These categories differ from the Game Ontology Project in that it is more focussed on the type of game as opposed to the inner mechanics of the game.

% Game design patterns
A similar approach is used by Bj\"ork et al \cite{Bjork2003, Bjork2003b} where game genres are used to derive game design patterns consisting of combinations of game elements. Game elements are seen as components that are used often within certain genres. The components are categorised into four distinct categories: holistic, bounding, temporal and structural. Holistic components are the elements that define how the game differs from other activities and how players can join a session and how the game session ends. Such components include the game instance, game and play session, set-up and set-down activities and extra-game activities. Unlike holistic components, bounding components describe the purpose of the game and what activities are allowed within the game context, these include the rules, the goals and sub-goals and the modes of play. The temporal components are those elements that divide the complete game session into smaller sessions such as the individual actions, events, closure or the end of 
such a session and the evaluation at the end. The structural components are the components that are interacted with and manipulated by the players. This category includes the players and their avatars, the interface and game elements. The terminology is different from the game elements as discussed before, which overlaps with the components as described by Bjork et al. Here the game elements represent entities within the game context that inform the players about the current game state.


%% Game contexts
% Magic circle of play
As the literature shows, game elements are usually defined as being design patterns used in game design. The game elements are elements that can be found in game contexts, although the definition of the game context is different depending on the task of the ontology. When classifying games, the game elements that are used are based on different genres, while for game design, the game elements are used as experience-design patterns. There seems to be no definite set of game elements for all purposes as some of the elements change context determined by the task of the ontology. So to define a set of game elements, it is important to first understand what constitutes as a game context. 

\subsection{Game Context}

In Huizinga's ``Homo Ludens: A Study of the Play-Element in Culture'' \cite{Huizinga} the play context is described as being a ``Magic'' circle. The magic circle is a concept that is used, in a similar fashion to assemblage theory \cite{Taylor2009}, to denote all elements within a certain contextual domain. The circle refers to a contextual domain that is either materially or ideally marked off and contains all elements that are relevant to the domain. The strength of this concept lies in that it allows for any domain to be viewed as being a circle for play. While Huizinga refers to this as a circle for play, the concept was later applied to digital games by Salen and Zimmerman \cite{Zimmerman2004}. In the context of a game, the magic circle is the domain that is entered when a player starts a game and thus all elements that are relevant during the game are the game elements of that game.

This notion of a magic circle is vastly different than the view of games as being interactive narratives \cite{Hartsook2011,Marchiori2011,Li2012,Jenkins2004}. Seeing games as interactive narratives creates the opportunity to compare games with more traditional story-telling such as writing, films and even static art. Many genres of games also overlap between traditional storytelling and games, such as interactive novels and role-playing games. However, not all types of games fall into this classification in a natural manner. When playing chess, not many people will be bothered with an actual story about warfare. Some of these games have a different type of narrative that breaks away from a traditional story and are more the narrative or experience of the game. For instance, tic-tac-toe does not have a proper story, but playing it creates a narrative between the players engaged in the game.

% Experiental Flow


\section{Learning Analytics}

There is an ongoing whish to increase study success of students in higher education. Stakeholders demand educational institutions to make data-driven decisions to improve perfomance \cite{Ferguson2012a}. Recently more and more focus is on identifying at-risk students to prevent drop-outs.

Now that the Web is globally used as a key course delivery method, many learning interactions of students no longer occur under the direct observation and influence of teachers \cite{Sheard2003}. This complicates decision making in classroom processes and makes it difficult to evaluate pedagogical strategies on their effectiveness \cite{Romero2007}. This is especially true for Massive Open Online Courses, since they have no face to face contact whatsoever(reference needed). Therefore educators need to find other ways to observe students and react to individual students' needs.

The introduction of virtual learning environments (VLE's)\footnote{Similair concepts are `Interactive Learning Environment', `Virtual Learning Environment', `Course Management System` and `Learning Management System'} and their ability of recording digital trails of students' learning actions resulted in the availability of large sets of data to education instutitions \cite{Ferguson2012a}\cite{Romero2007}. An increasing interest arises in analysing interaction data of the learning with the virtual learning environment automatically \cite{Muehlenbrock2005}. It is also not easy for the faculty to interpret large datasets of learning-related data. Furthermore, VLE's insufficiently support stakeholders with extracting, aggregating and visualising pedagogically useful information \cite{Zhang2007}\cite{Dawson2010}\cite{Zaiane2001}. At the same time it would be useful if the system could guide a student's learning process automatically by recommending online activities or resourses \cite{Zaiane2001}.
%% Maybe: EDM denotes the area that concerns itself with using Data Mining techniques to answer such questions, Academic analytics specialized this ....., Learning analytics uses similair techniques but tries to give constructive feedback to both teachers and students to help improve learners and the environment in which they learn.
Learning Analytics assists both educators and learners in this matter and is defined by the 1st International Conference on Learning Analytics and Knowledge as follows:
\begin{quote}
"Learning analytics is the measurement, collection, analysis and reporting of data about learners and their contexts, for purposes of understanding and optimising learning and the environments in which it occurs." \cite{lak2011}
\end{quote}

The field of learning analytics has its origins in the field of Educational Data Mining (EDM), which is "concerned with developing methods for exploring the unique types of data that come from educational settings, and using those methods to better understand students, and the settings which they learn in" \cite{Baker2009}. A comprehensive overview of this field is given by \cite{Romero2007}. In comparision to EDM, learning analytics is more focused on improving the situation while learning takes place. It is therefore more aiming at generating feedback to teachers and students or altering the learning environment automatically. Educational data mining is more open-ended analysis of data captured in education processes. 

Before the term learning analytics came in to play, the term academic analytics was coined by WebCT (now Blackboard) to describe their new feature. WebCT's academic analytics was a module inspired by business intelligence, which was a set of tools used to make data-driven decisions about a business. Academic analytics applied this to the business of higher education, and focused on enrollment optimization and student success. Stakeholders for academic analytics systems are historically at the institutional level that seek to make statistical predictions and identify key indicators. Learning analytics research typically attempts to answer different questions coming from different stakeholders. The Signals project from Perdue University, as described in \cite{Arnold2010}, is a well-known example of academic analytics where also students are informed of how they are doing using the traffic light metaphor. Their project lies in somewhere between academic analytics and learning analytics. Academic analytics 
differs from EDM in that it is more hypothesis driven. The term action analytics emerged to stress that academic analytics is nowadays mostly focused on actionable analytics (i.e. having it early enough so that something could still be done).

\paragraph{Academic Analytics}
\begin{itemize}
  \item \cite{Campbell2007} ....
  \item \cite{Baepler2010}: Besides educational data mining, the field of academic analytics also emerged as a method to make more data-driven decisions . Academic analytics differs from EDM in that it is more hypothesis driven. The term action analytics emerged to stress that academic analytics is mostly focused on actionable analytics (i.e. having it early enough so that something could still be done).
\end{itemize}
\paragraph{Personalized Learning}
Introduceer de notie van personalized learning, met een citaat uit een paper. 
\begin{itemize}
  \item \cite{Weber2001}: Work has been done on Adaptive hypermedia systems that can integrate external resources (open course).
  \item Brusilovsky2001: Static hypermedia systems suffer from the inability to adapt to specific users' knowledge, needs and preferences. Adaptive hypermedia systems combine the fields of hypertext and user modelling to overcome this.
%  \item Arruabarrena2002: On evaluating adaptive educational systems
  \item Brusilovsky2007: Access to online educational is not a problem anymore, however personalised access is.
%  \item Romero2005: 
  \item \cite{Brusilovsky2003}
  \item \cite{Henze2001}
  \item \cite{Farzan2004}
  \item \cite{Tand2005}
  \item \cite{Modritscher2011}
\end{itemize}

\paragraph{Social Learning}
\begin{itemize}
  \item \cite{Shum2011}
  \item \cite{Ullmann2011}
  \item \cite{Haiming2012}
  \item \cite{Fournier2011}
  \item \cite{Ferguson2012b}
\end{itemize}

Social Learning Analytics:


%Large amounts of students that are studying with a decreasing amount of contact hours can cause problems for a teacher that wants to keep track of progress in the group and wants to adjust his teaching methods to specific learner's needs. At the same time, we observe that the learning increasingly takes place in digital environments that are able to track students and have the potential of interacting on a personalized level due to the fact that each student runs his own instance of such systems. The enourmous amounts of data that are gathered in this way can be analysed automatically and used to base system actions on.

%Teachers are typically not able to analyse the large amounts of data themselves, however they can act upon summarized information that can be supplied by such a system. Teachers are able to give specific aid to a few struggling students, if they would know which students are struggling.

%Teachers also typically have difficulties with identifying problematic components in their suggested curriculum. These problematic components are sometimes visible to students, but the feedback resulting from that to the teacher is not always sufficient. Sometimes these problematic components are not even that clear to students, since they could also explain their struggling with other factors.

%Students are also affected by the reduction in personal education due to student increase. They get less feedback on their own progress and also have the risk of being stuck too long because they are not directed to other resources to get them over the edge.

%Using the large amounts of data to give feedback on the learning process to both teachers and students is called Learning Analytics.

%\subsection{Introduction, Definition, Subcategories}
%According to the 1st International Conference on Learning Analytics and Knowledge:
%\begin{quote}
%"Learning analytics is the measurement, collection, analysis and reporting of data about learners and their contexts, for purposes of understanding and optimising learning and the environments in which it occurs." \cite{lak2011}
%%\end{quote}

\newpage
Learning Analytics differs from Academic Analytics in the intended stakeholders and their questions. 

Where academic analytics tries to answer questions about performance on an institutional, regional, 

disambiguate:
Learning Analytics
Academic Analytics: business intelligence on higher education. On institutional, regional and (inter)national level analysis of perfomance and learner profiles. Comparison between institutions. (For administrators, funders, marketing and governments/authorities)
Action Analytics: 
Visual Analytics





\cite{Ferguson2012a} \cite{Greller2012}
\begin{itemize}
\item Position in field: Educational Data Mining, Academic Analytics.
\item Introduction of the EDM tradition
\item Introduction of the Academic Analytics tradition
\item Differentiating Learning Analytics from EDM and Academic Analytics
\end{itemize}
\subsection{Personalised learning}
\begin{itemize}
\item Personalised learning: Student modelling, Recommendation systems, Adaptive systems
\item The connection between Learning Analytics and modelling students’ knowledge
\item The connection between Learning Analytics and recommender systems
\item The connection between Learning Analytics and adaptive systems
\end{itemize}
\subsection{Learning as social activity}
\begin{itemize}
\item Learning as a social activity: Social Learning Analytics, Networked Learning, Collaborative learning, E-mentoring
\end{itemize}
\subsection{Results}
\begin{itemize}
\item What improvements have been shown?
\end{itemize}


\section{Gamification and Learning Analytics applied on Education}



\section{Conclusion}

\section{Discussion}


\pagebreak
\bibliography{literature,literature2}
\bibliographystyle{plain}

\end{document}
