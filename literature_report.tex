\documentclass[10pt]{article}

% Math formatting
\usepackage{amsmath}
\usepackage{amssymb}




% Lay out packages
%\usepackage[margin=3.5cm]{geometry}
\usepackage[utf8]{inputenc}
%\usepackage{mathpazo}

% Dutch style of paragraph formatting, i.e. no indents. 
\setlength{\parskip}{1.3ex plus 0.2ex minus 0.2ex}
\setlength{\parindent}{0pt}

% Command for Horizontal lines
\newcommand{\HRule}{\rule{\linewidth}{0.5mm}}
% Command for degree symbol
\newcommand{\degree}{\ensuremath{^\circ}}

% Add images and pdf pages
\usepackage{graphicx}
\usepackage{pdfpages}



% Colored rows in tables
\usepackage[table]{xcolor}

% Clickable links with hyperref package
\usepackage[pdfborder={0 0 0 0}, linkcolor=black, urlcolor=blue]{hyperref}

% Fancy Header
\usepackage{fancyhdr}
\pagestyle{fancy}

% Psuedo code
\usepackage{algorithm}
\usepackage{algorithmic}

\rhead{\large\bfseries Deliverable 1: Literature study} % right header: document title
\lhead{\textsc{Jurriaans}, \textsc{Latour}} % left header: document author
\cfoot{\large \thepage} % center footer: page number

\setlength{\headheight}{18pt}

\begin{document}

\begin{titlepage}
\begin{center}
\includegraphics[width=1\textwidth]{img/uva}\\[1cm]
\HRule \\[0.4cm]
% Title
 Deliverable 1: Literature study \\\small \small{[Version 0.1]}\\[0.4cm]
\HRule \\[1cm]
\begin{tabular*}{0.95\textwidth}{@{\extracolsep{\fill}} l r}
Robrecht \textsc{Jurriaans} & Sander \textsc{Latour} \\
\textsc{5887380} & \textsc{} \\
\end{tabular*}



\vfill \today
\end{center}
\end{titlepage}

\tableofcontents
\pagebreak


\section{Introduction}
% Introduce the current status of education and the problems that lie with motivation
There is an ongoing desire to improve study success in higher education.
Work has been done to identify students that have a high risk of dropping out. (refs)

Two factors in study success seem to be 1) the match between the level of the student and the level of the courses' materials and 2) the student's engagement and motivation.

Students often come from different backgrounds and 

\cite{Horizon2012}

% Use psychological evidence to determine key-factors in motivation and how it improves learning


% Introduce Gamification and Learning Analytics as solutions to this motivation problem



% Research Questions


% Overview of Paper

\section{Gamification}


\subsection{Definition}
% Gamification definition, Deterding 2011
% Make distinction between fun in learning; serious games; game design in non-gaming environments


\subsection{Background}
% History of Motivation and Incentive Systems, Pearson 2012


\subsection{Gamification Methods}
% Gamification Methods

\subsubsection{Game Design Patterns and Mechanics}
% Game design patterns and mechanics

\subsubsection{Game Design Principles and Heuristics}
% Game design principles and heuristics

\subsubsection{Game Models}
% Game Models


\subsection{Applications and Frameworks}
% Applications and Existing Frameworks
% Mainly address what these frameworks introduce/use and what factors were important in the success of the application


  \subsection{Open Issues}
% Open issues in Gamification
% Future Work (Narrative and experiences)


\section{Learning Analytics}

There is an ongoing whish to increase study success of students in higher education. Stakeholders demand educational institutions to make data-driven decisions to improve perfomance \cite{Ferguson2012a}. Recently more and more focus is on identifying at-risk students to prevent drop-outs.

Now that the Web is globally used as a key component of course delivery, many learning interactions of students no longer occur under the direct observation and influence of teachers \cite{Sheard2003}. This complicates decision making in classroom processes and makes it difficult to evaluate pedagogical strategies on their effectiveness \cite{Romero2007}. This is especially true for Massive Open Online Courses, since they have no face to face contact whatsoever(reference needed). Therefore educators need to find other ways to observe students and react to individual students' needs.

The introduction of virtual learning environments (VLE's)\footnote{Similair concepts are `Interactive Learning Environment', `Course Management System` and `Learning Management System'} and their ability of recording digital trails of students' learning actions resulted in the availability of large sets of data to education instutitions \cite{Ferguson2012a}\cite{Romero2007}. An increasing interest arises in analysing interaction data of the learning with the virtual learning environment automatically \cite{Muehlenbrock2005}. It is also not easy for the faculty to interpret large datasets of learning-related data. Furthermore, VLE's insufficiently support stakeholders with extracting, aggregating and visualising pedagogically useful information \cite{Zhang2007}\cite{Dawson2010}\cite{Zaiane2001}. At the same time it would be useful if the system could guide a student's learning process automatically by recommending online activities or resourses \cite{Zaiane2001}.
%% Maybe: EDM denotes the area that concerns itself with using Data Mining techniques to answer such questions, Academic analytics specialized this ....., Learning analytics uses similair techniques but tries to give constructive feedback to both teachers and students to help improve learners and the environment in which they learn.
Learning Analytics assists both educators and learners in this matter and is defined by the 1st International Conference on Learning Analytics and Knowledge as follows:
\begin{quote}
"Learning analytics is the measurement, collection, analysis and reporting of data about learners and their contexts, for purposes of understanding and optimising learning and the environments in which it occurs." \cite{lak2011}
\end{quote}

The field of learning analytics has its origins in the field of Educational Data Mining, which is "concerned with developing methods for exploring the unique types of data that come from educational settings, and using those methods to better understand students, and the settings which they learn in" \cite{Baker2009}. A comprehensive overview of this field is given by \cite{Romero2007}. Besides learning analytics, also the related field of academic analytics originated from EDM. Acadamic analytics 

\textit{Also Academic Analytics was derived from EDM. Say how LA, AA and EDM differ from each other. Two future goals, a.o., can be deduced from publications in Learning Analytics: Personalized learning and Social Learning}

\paragraph{Academic Analytics}
\begin{itemize}
  \item \cite{Campbell2007} ....
  \item \cite{Baepler2010}: Besides educational data mining, the field of academic analytics also emerged as a method to make more data-driven decisions . Academic analytics differs from EDM in that it is more hypothesis driven. The term action analytics emerged to stress that academic analytics is mostly focused on actionable analytics (i.e. having it early enough so that something could still be done).
\end{itemize}
\paragraph{Personalized Learning}
Introduceer de notie van personalized learning, met een citaat uit een paper. 
\begin{itemize}
  \item \cite{Weber2001}: Work has been done on Adaptive hypermedia systems that can integrate external resources (open course).
  \item Brusilovsky2001: Static hypermedia systems suffer from the inability to adapt to specific users' knowledge, needs and preferences. Adaptive hypermedia systems combine the fields of hypertext and user modelling to overcome this.
%  \item Arruabarrena2002: On evaluating adaptive educational systems
  \item Brusilovsky2007: Access to online educational is not a problem anymore, however personalised access is.
%  \item Romero2005: 
  \item \cite{Brusilovsky2003}
  \item \cite{Henze2001}
  \item \cite{Farzan2004}
  \item \cite{Tand2005}
  \item \cite{Modritscher2011}
\end{itemize}

\paragraph{Social Learning}
\begin{itemize}
  \item \cite{Shum2001 
  \item \cite{Ullmann2011}
  \item \cite{Haiming2012}
  \item \cite{Fournier2011}
  \item \cite{Ferguson2012b}
\end{itemize}

Social Learning Analytics:


%Large amounts of students that are studying with a decreasing amount of contact hours can cause problems for a teacher that wants to keep track of progress in the group and wants to adjust his teaching methods to specific learner's needs. At the same time, we observe that the learning increasingly takes place in digital environments that are able to track students and have the potential of interacting on a personalized level due to the fact that each student runs his own instance of such systems. The enourmous amounts of data that are gathered in this way can be analysed automatically and used to base system actions on.

%Teachers are typically not able to analyse the large amounts of data themselves, however they can act upon summarized information that can be supplied by such a system. Teachers are able to give specific aid to a few struggling students, if they would know which students are struggling.

%Teachers also typically have difficulties with identifying problematic components in their suggested curriculum. These problematic components are sometimes visible to students, but the feedback resulting from that to the teacher is not always sufficient. Sometimes these problematic components are not even that clear to students, since they could also explain their struggling with other factors.

%Students are also affected by the reduction in personal education due to student increase. They get less feedback on their own progress and also have the risk of being stuck too long because they are not directed to other resources to get them over the edge.

%Using the large amounts of data to give feedback on the learning process to both teachers and students is called Learning Analytics.

%\subsection{Introduction, Definition, Subcategories}
%According to the 1st International Conference on Learning Analytics and Knowledge:
%\begin{quote}
%"Learning analytics is the measurement, collection, analysis and reporting of data about learners and their contexts, for purposes of understanding and optimising learning and the environments in which it occurs." \cite{lak2011}
%%\end{quote}

\newpage
Learning Analytics differs from Academic Analytics in the intended stakeholders and their questions. 

Where academic analytics tries to answer questions about performance on an institutional, regional, 

disambiguate:
Learning Analytics
Academic Analytics: business intelligence on higher education. On institutional, regional and (inter)national level analysis of perfomance and learner profiles. Comparison between institutions. (For administrators, funders, marketing and governments/authorities)
Action Analytics: 
Visual Analytics





\cite{Ferguson2012a} \cite{Greller2012}
\begin{itemize}
\item Position in field: Educational Data Mining, Academic Analytics.
\item Introduction of the EDM tradition
\item Introduction of the Academic Analytics tradition
\item Differentiating Learning Analytics from EDM and Academic Analytics
\end{itemize}
\subsection{Personalised learning}
\begin{itemize}
\item Personalised learning: Student modelling, Recommendation systems, Adaptive systems
\item The connection between Learning Analytics and modelling students’ knowledge
\item The connection between Learning Analytics and recommender systems
\item The connection between Learning Analytics and adaptive systems
\end{itemize}
\subsection{Learning as social activity}
\begin{itemize}
\item Learning as a social activity: Social Learning Analytics, Networked Learning, Collaborative learning, E-mentoring
\end{itemize}
\subsection{Results}
\begin{itemize}
\item What improvements have been shown?
\end{itemize}


\section{Gamification and Learning Analytics applied on Education}

\section{Conclusion}

\section{Discussion}


\pagebreak
\bibliography{literature}
\bibliographystyle{plain}

\end{document}
