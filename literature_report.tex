\documentclass[10pt]{article}

% Math formatting
\usepackage{amsmath}
\usepackage{amssymb}




% Lay out packages
%\usepackage[margin=3.5cm]{geometry}
\usepackage[utf8]{inputenc}
%\usepackage{mathpazo}

% Dutch style of paragraph formatting, i.e. no indents. 
\setlength{\parskip}{1.3ex plus 0.2ex minus 0.2ex}
\setlength{\parindent}{0pt}

% Command for Horizontal lines
\newcommand{\HRule}{\rule{\linewidth}{0.5mm}}
% Command for degree symbol
\newcommand{\degree}{\ensuremath{^\circ}}

% Add images and pdf pages
\usepackage{graphicx}
\usepackage{pdfpages}



% Colored rows in tables
\usepackage[table]{xcolor}

% Clickable links with hyperref package
\usepackage[pdfborder={0 0 0 0}, linkcolor=black, urlcolor=blue]{hyperref}

% Fancy Header
\usepackage{fancyhdr}
\pagestyle{fancy}

% Psuedo code
\usepackage{algorithm}
\usepackage{algorithmic}

\rhead{\large\bfseries Deliverable 1: Literature study} % right header: document title
\lhead{\textsc{Jurriaans}, \textsc{Latour}} % left header: document author
\cfoot{\large \thepage} % center footer: page number

\setlength{\headheight}{18pt}

\begin{document}

\begin{titlepage}
\begin{center}
\includegraphics[width=1\textwidth]{img/uva}\\[1cm]
\HRule \\[0.4cm]
% Title
 Deliverable 1: Literature study \\\small \small{[Version 0.1]}\\[0.4cm]
\HRule \\[1cm]
\begin{tabular*}{0.95\textwidth}{@{\extracolsep{\fill}} l r}
Robrecht \textsc{Jurriaans} & Sander \textsc{Latour} \\
\textsc{5887380} & \textsc{} \\
\end{tabular*}



\vfill \today
\end{center}
\end{titlepage}

\tableofcontents
\pagebreak


\section{Introduction}
% Introduce the current status of education and the problems that lie with motivation

\cite{Horizon2012}

% Use psychological evidence to determine key-factors in motivation and how it improves learning


% Introduce Gamification and Learning Analytics as solutions to this motivation problem



% Research Questions


% Overview of Paper

\section{Gamification}


\subsection{Definition}
% Gamification definition, Deterding 2011
% Make distinction between fun in learning; serious games; game design in non-gaming environments


\subsection{Background}
% History of Motivation and Incentive Systems, Pearson 2012


\subsection{Gamification Methods}
% Gamification Methods

\subsubsection{Game Design Patterns and Mechanics}
% Game design patterns and mechanics

\subsubsection{Game Design Principles and Heuristics}
% Game design principles and heuristics

\subsubsection{Game Models}
% Game Models


\subsection{Applications and Frameworks}
% Applications and Existing Frameworks
% Mainly address what these frameworks introduce/use and what factors were important in the success of the application


\subsection{Open Issues}
% Open issues in Gamification
% Future Work (Narrative and experiences)




\section{Learning Analytics}
\begin{itemize}
\item Learning Analytics (definition, introduction)
\item Position in field: Educational Data Mining, Academic Analytics.
\item Introduction of the EDM tradition
\item Introduction of the Academic Analytics tradition
\item Differentiating Learning Analytics from EDM and Academic Analytics
\item Personalised learning: Student modelling, Recommendation systems, Adaptive systems
\item The connection between Learning Analytics and modelling students’ knowledge
\item The connection between Learning Analytics and recommender systems
\item The connection between Learning Analytics and adaptive systems
\item Learning as a social activity: Social Learning Analytics, Networked Learning, Collaborative learning, E-mentoring
\item What improvements have been shown?
\end{itemize}


\section{Gamification and Learning Analytics applied on Education}

\section{Conclusion}

\section{Discussion}


\pagebreak
\bibliography{literature}
\bibliographystyle{plain}

\end{document}
